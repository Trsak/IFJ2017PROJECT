\documentclass[11pt,a4paper]{article}

\usepackage[utf8]{inputenc}
\usepackage{times}
\usepackage[left=2cm,top=3cm,text={17cm, 24cm}]{geometry}
\usepackage[czech]{babel}
\usepackage{graphicx}
\usepackage{setspace}
\usepackage{hyperref}
\usepackage{multirow}
\usepackage{pdflscape}

\begin{document}

    \begin{titlepage}
    \begin{center}
      {\Huge
      \textsc{Vysoké učení technické v~Brně} \\
      \medskip
		\huge{\textsc{Fakulta informačních technologií}}
       }\\
      %\vspace{\stretch{0.191}}
      \begin{figure}[h]
		\begin{center}
		\scalebox{0.8}{\includegraphics{fit.png}}
		\end{center}
	  \end{figure}
	  %\vspace{\stretch{0.191}}
      {\LARGE
      Dokumentace k projektu do předmětů IFJ a IAL}\\
      \LARGE{\textbf{Implementace překladače imperativního jazyka IFJ17}}
      \vspace{\stretch{0.060}}
      
      {\LARGE Tým 065, varianta I}
      
      {\LARGE 31. března 2017}
      \vspace{\stretch{0.618}}
      
      {\LARGE Seznam autorů:\\}
      {\Large\itshape Vedoucí: Bártl Roman (xbartl06) - 25\%\par}
        {\Large\itshape Bartošek Jan (xbarto92) - 25\%\par}
        {\Large\itshape Odehnal Tomáš (xodehn08) - 25\%\par}
        {\Large\itshape Šopf Petr (xsopfp00) - 25\%\par}
        \vspace{\stretch{0.718}}
        {\Large \textbf{Rozšíření:} UNARY, BASE, IFTHEN}
        \vspace{\stretch{0.2}}
    \end{center}
\end{titlepage}
    
    \doublespacing
	\tableofcontents
	\singlespacing
    \newpage

\addtocontents{toc}{\setcounter{tocdepth}{3}}
\section{Scanner}
Implementoval xbarto92

	\subsection{Postup Implementace}
	Při prvním návrhu lexikální analýzy jsem se inspiroval ukázkovým příkladem, kde byl scanner řešený s využitím přepínače switch umístěného v nekonečné smyčce. První verze scanneru spatřila světlo světa již 7. října, ovšem její funkčnost byla naprosto nevyhovující, a tak jsem scanner postupně opravoval a rozšiřoval.
	
	Původní verze scanneru vracela pouze integer hodnotu, která reprezentovala přečtený lexém a nedokázala rozlišovat klíčová slova od identifikátorů. Tento problém jsem vyřešil vytvořením pole, jehož hodnoty odpovídají všem rezervovaným klíčovým slovům a v případě zjištění identifikátoru, pak jeho porovnáním.
	
	Následovalo vytvoření struktury reprezentující token nesoucí informace o typu přečteného lexému (číslo, ID, atd..), jeho hodnotu (např.: název proměnné) a v poslední řadě zde byla přidána i hodnota řádku, která slouží pro výpis chybových hlášek a lepší debug. Dřívější myšlenka byla taková, že scanner by měl hodnotu přečteného tokenu hned zapsat do tabulky symbolů. Z tohoto důvodu jsem připravil binární strom a dynamický stack, právě pro práci s touto tabulkou. Nicméně po delší úvaze jsme se rozhodli zanechat tuto úlohu mimo parser, a tak se jejím vyřešením zaobírali mí kolegové.
	
	V momentě, kdy scanner již obstojně interpretoval jakýkoliv příchozí lexém jsem se začal zaobírat rozšířeními UNARY a BASE, kde první z nich jsem jako speciální token posílal ke zpracování parseru, a ten druhý nahrál a funkcí strtol převedl hodnotu čísla do dekadické soustavy a posílal jako jakékoli jiné číslo. V neposlední řadě jsem se zaobíral testováním všech možných chybových stavů a jejich správným řešením.

	\subsection{Funkce}
	Lexikální analýza implementovaná v souborech scanner.c a scanner.h cyklicky načítá ze vstupu znak po znaku a každý ihned interpretuje a kontroluje jejich vzájemnou návaznost.
	
	Funkce getNextToken(), která obstarává celou lexikální analýzu, se volá v parseru při potřebě následujícího tokenu. Při správné posloupnosti znaků je vrácen token, neboli struktura, která se skládá ze tří hodnot (viz výše). V opačném případě se na chybový výstup vypíše odpovídající chybová hláška spolu s řádkem v kódu, na kterém k chybě došlo.
    \newpage

\subsection{Konečný Automat Lexikální Analýzy}
\vspace{1cm}
	\centerline{\includegraphics[height=21cm]{KA.png}}
    {\newpage}



\section{Parser - Syntax}
    Implementoval xbartl06

    \subsection{Implementace}
    \noindent Hinzufügen den Text hier!

    \subsection{Funkce}
    \noindent Hinzufügen den Text hier!
    \newpage



\section{Parser - Sémantika}
    Implementoval xodehn08

    \subsection{Postup Implementace}
    \noindent Hinzufügen den Text hier!

    \subsection{Funkce}
    \noindent Hinzufügen den Text hier!
    \newpage



\section{Generátor}
    Implementoval xsopfp00

    \subsection{Postup Implementace}
    \noindent Hinzufügen den Text hier!

    \subsection{Funkce}
    \noindent Hinzufügen den Text hier!
    \newpage
\end{document}